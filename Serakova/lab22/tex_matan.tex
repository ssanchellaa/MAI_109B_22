\documentclass[a5paper]{book}
\usepackage[english, russian]{babel}
\usepackage[utf8]{inputenc} % Required for inserting images
\usepackage{amssymb}
\usepackage[9pt]{extsizes}
\usepackage[tracking=true]{microtype}

\usepackage{amsfonts}
\usepackage{amsmath}
\usepackage[left=15mm, top=15mm, right=15mm, bottom=15mm, nohead, nofoot]{geometry}

\usepackage{ragged2e}
\setcounter{page}{170}


\begin{document}
\parindent = 0.8cm

\begin{center}
\hspace{30pt}
{\scriptsize\centering\textsection ГЛ. V. ДИФФЕРЕНЦИРОВАНИЕ ФУНКЦИЙ ОДНОЙ ПЕРЕМЕННОЙ}
\hspace{15pt}
\textbf{[96}\\

\end{center}
	
\smallskip
	
Например, если 
$$
y=\frac{1}{2}\ x^4-\frac{1}{6}\ x^8+2x^2+\frac{4}{3}\ x-\frac{1}{2}\,
$$
\noindent то
$$
y^{\prime}=2x^3-\frac{1}{2}\ x^2+4x+\frac{4}{3}\, y^{\prime\prime}=6x^2-x+4,
$$
$$
y^{\prime\prime\prime}=12x-1, y^{\prime\prime\prime\prime}=12,
$$
\noindent так что все последующие производные равны тождественно нулю. Или Пусть
$$
y=\ln (x+\sqrt{x^2+1});
$$
\noindent тогда
$$
y^\prime=\dfrac{1}{\sqrt{x^2+1}}, y^{\prime\prime}=\dfrac{x}{(x^2+1)^{^3/_2}}, y^{\prime\prime\prime}=\dfrac{2x^2-1}{(x^2+1)^{^5/_2}} и т.д.
$$
Заметим, что по отношению к производным высших порядков так же, индуктивно, можно установить понятие \textls{односторонней} производной [ср. $n^{\circ}86$]. Если функция $y=f(x)$ определена лишь в некотором промежутке \slshape X \upshape, то, говоря о производной любого порядка на \textls{конце} его, всегда имеют в виду именно одностороннюю производную.

\smallskip
	
\textbf{96. Общие формулы для производных любого порядка.} Итак, для того, чтобы вычислить n-ю производную от какой-либо функции, вообще говоря, нужно предварительно вычислить производные всех предшествующих порядков. Однако в ряде случаев удается установить такое общее выражение для n-й производной, которое зависит непосредственно от n и не содержит более обозначений предшествующих производных.

При выводе таких общих выражений иногда бывают полезны формулы:
$$
(cu)^{(n)}=c\cdot u^{(n)}, (u\pm v)^{(n)}=u^{(n)}\pm v^{(n)},
$$
обобщающие на случай высших производных известные читателю правила I и II $n^{\circ}83$. Их легко получить последовательным применением этих правил.

1) Рассмотрим сначала степенную функцию $y=x^{\mu}$, где $\mu$-любое вещественное число. Имеем последовательно:
$$
y^\prime =\mu x^{\mu-1}, y^{\prime\prime}=\mu(\mu-1)x^{\mu-2},
$$
$$
y^{\prime\prime\prime}=\mu(\mu-1)(\mu-2)x^{\mu-3},\ldots
$$

\noindent Легко усмотреть отсюда и общий закон:
$$
y^{(n)}=\mu(\mu-1)\ldots(\mu-n+1)x^{\mu-n},
$$
\noindent который доказывается по методу математической индукции.

Если, например, взять $\mu$=--1, то получим
$$
(\frac{1}{x})^(n)=(-1)(-2)\ldots(-n)x^{-1-n}=\dfrac{(-1)^n\cdot n!}{x^{n+1}}.
$$.

\newpage

	{\normalsize
 \noindent\textbf{96]}
 \hspace{15pt}
\scriptsize\centering \S \ 3.\ \footnotesizeПРОИЗВОДНЫЕ И ДИФФЕРЕНЦИАЛЫ ВЫСШИХ ПОРЯДКОВ}
 

\setlength{\parskip}{0.3cm}	
\smallskip	
Когда само $\mu$ есть \textls{натуральное} число $m$, то $m$-я производная от $x^{m}$ будет уже постоянным числом $m!$, а все следующие - нулями. Отсюда ясно, что и для целого многочлена степени $m$имеет место аналогичное обстоятельство.

2)Пусть теперь $y=\ln x$. Прежде всего, имеем
$$
y^\prime=(\ln x)^\prime=\dfrac{1}{x}.
$$

Возьмем отсюда производную $(n-1)$-го порядка по соответствующей формуле из 1), заменив в ней $n$ на $n-1$; мы и получим тогда
$$
y^{(n)}=(y^\prime)^{(n-1)}=(\frac{1}{x})^{(n-1)}=\dfrac{(-1)^{n-1}(n-1)!}{x^n}.
$$

3) Если $y=a^x$, то
$$
y^\prime=a^x\cdot\ln a, y^{\prime\prime}=a^x\cdot(\ln a)^2,\ldots
$$
\noindent Общая формула
$$
y^(n)=a^x\cdot(\ln a)^n
$$ 
\noindent легко доказывается по методу математической индукции.

В частности, очевидно,
$$
(\exp ^x)^{(n)}=\exp ^x.
$$ 

4) Положим $y=\sin x$; тогда
$$
y^\prime=\cos x,  y^{\prime\prime}= -\sin x,  y^{\prime\prime\prime}=---\cos x,
$$
$$
y^{\prime\prime\prime\prime}=\sin x,  y^{(5)}=\cos x,\ldots
$$

\noindent На этом пути найти требуемое \textls{общее} выражение для $n$-й производной трудно. Но дело сразу упрощается, если переписать формулу для первой производной в виде $y^{\prime}= \sin(x+\frac{\pi }{2})$; становится ясным, что при каждом дифференцировании к аргументу будет прибавляться $\frac{\pi }{2}$, так что
$$
(\sin x)^{(n)}=\sin (x+n\cdot\frac{\pi }{2}).
$$
\noindent Аналогично получается и формула
$$
(\sin x)^{(n)}=\sin (x+n\cdot\frac{\pi }{2}).
$$
\newpage

\begin{center}
\hspace{30pt}
{\scriptsize\centering\textsection ГЛ. V. ДИФФЕРЕНЦИРОВАНИЕ ФУНКЦИЙ ОДНОЙ ПЕРЕМЕННОЙ}
\hspace{15pt}
\textbf{[97}\\

\end{center}

5) Остановимся еще на функции $y=\arctan{x}$. Поставим себе задачей выразить $y^{n}$ через $y$. Так как $x=\tan{y}$, то
$$
y^{\prime}=\dfrac{1}{1+x^2}=\cos{y}\cdot\sin{(y+\frac{\pi}{2})}.
$$
\noindent Дифференцируя вторично по $x$ (и помня, что $y$ есть функция от $x$), получим
$$
y^{\prime\prime}=\left[ {-\sin{y}\cdot\sin(y+\frac{\pi}{2})+\cos{y}\cdot\cos{(y+\frac{\pi}{2})}} \right]\cdot y^{\prime}=
$$
$$=\cos^2{y}\cdot\cos{(2y+\frac{\pi}{2})}=\cos^2{y}\cdot\sin 2(2y+\frac{\pi}{2}).
$$
\noindent Следующее дифференцирование дает:
$$
y^{\prime\prime\prime}=\left[ {-2\sin{y}\cdot\cos{y}2\sin(y+\frac{\pi}{2})+2\cos^2{y}\cdot\cos 2(y+\frac{\pi}{2})} \right]\cdot y^{\prime}=
$$
$$
=2\cos^3{y}\cdot\cos{(3y+2\cdot\frac{\pi}{2})}=2\cos^3{y}\cdot\sin 3(2y+\frac{\pi}{2}).
$$

Общая формула
$$
y^{(n)}=(n-1)!\cos^{n}y\cdot\sin n(y+\frac{\pi}{2})
$$
\noindent оправдывается по методу математической индукции.

\textbf{97. Формула Лейбница.} Как мы заметили в начале предыдущего номера, правила I и II $n^{\circ}83$ непосредственно переносятся и на случай производных любого порядка. Сложнее обстоит дело с правилом III, относящимся к дифференцированию произведения.
Предположим, что функции $u$, $\upsilon$ от $x$ имеют каждая в отдельности производные до $n$-го порядка включительно; докажем, что тогда их произведение $y=u\upsilon$ также имеет $n$-ю производную, найдем ее выражение.
Станем, применяя правило III, последовательно дифференцировать это произведение; мы найдем:
$$
y^{\prime}=u^{\prime}\upsilon + u\upsilon^{\prime},{  } 
  y^{\prime\prime}=u^{\prime\prime}\upsilon +2u^{\prime}\upsilon^{\prime}+u\upsilon^{\prime\prime},
$$
$$
y^{\prime\prime\prime}=u^{\prime\prime\prime}\upsilon +3u^{\prime\prime}\upsilon^{\prime}+3u^{\prime}\upsilon^{\prime\prime}+u\upsilon^{\prime\prime\prime},\ldots
$$
Легко подметить закон, по которому построены все эти формулы: правые части их напоминают разложение стпеней бинома: $(u+\upsilon)^{2}$, $(u+\upsilon)^{3},\ldots,$ лишь вместо степеней $u, \upsilon$ стоят производные соответствующих порядков. Сходство станет более полным, если

\end{document}