\documentclass[a5paper,18pt]{book}
\usepackage[english, russian]{babel}
\usepackage[utf8]{inputenc} % Required for inserting images
\usepackage{amssymb}

\usepackage{amsfonts}
\usepackage{amsmath}
\usepackage[left=10mm, top=15mm, right=20mm, bottom=15mm, nohead, nofoot]{geometry}
\setcounter{page}{288}

\begin{document}
 \begin{center}
     
     ГЛ. V. ДИФФЕРЕНЦИАЛЬНОЕ ИСЧИСЛЕНИЕ
     \par\noindent\rule{\textwidth}{1.1pt} %the line after the title
     
 \end{center}
  \par  \[\int \frac{1}{1+x^2}\, dx=\bigg\{ 
  \begin{array}{lr}
       & \textrm{arctg}\, x+c, \\
       & -\textrm{arctg}\, x+\tilde{c},
  \end{array}
   \]
 \[\int \textrm{sh}\, x\, dx=\textrm{ch}\, x+c,\]
 \[\int \textrm{ch} \, x\, dx=\textrm{sh}\, x+c,\]
 \[\int \frac{1}{\textrm{ch}^2\, x}\, dx=\textrm{th} \, x+c,\]
 \[\int \frac{1}{\textrm{sh}^2\, x}\, dx=-\textrm{cth} \, x+c,\]
  \[\int \frac{1}{\sqrt{x^2 \pm 1}}\, dx=\ln{\big|x+\sqrt{x^2 \pm 1}\big|}+c,\]
  \[\int \frac{1}{1-x^2 }\, dx=\frac{1}{2}\ln{\big|\frac{1+x}{1-x}\big|}+c.\]
  \parКаждая из этих формул рассматривается на текущих промежутках вещественной оси $\mathbb{R}$,на которых определена соответсвующая подынтегральная функция.Если таких промежутков несколько,то постоянная c в правой части может меняться от промежутка к промежутку.
  \par Рассмотрим теперь некоторые примеры,показывающие соотношения (5),(6) и (7) в работе.
  \par Сделаем предварительно следующее общее замечание.
  \par Поскольку, найдя одну какую-нибудь первообразную заданной на промежутке функции, остальные можно получить добавлением постоянных, то условимся для сокращения записи всюду в дальнейшем произвольную постоянную добавлять только к окнчательному результату, представляющему из себя конкретную первообразную данной функции.
  \par \textbf{a. Линейность неопределенного интеграла.} Этот заголовок должен означать, что в силу соотношения (5) первообразную от линейной комбинаций функций можно искать как линейную комбинацию первообразных этих функций.
  \par Пример 3.
  \[\int(a_0+ a_1 x+...+a_nx^{n})\, dx=a_0\int 1\, dx + a_1\int x\, dx +...+a_n \int x^{n}\, dx =\]
  \[=c+a_0x+\frac{1}{2}a_1x^2+...+\frac{1}{n+1}a_n x^{n+1}\]
  \newpage
  \begin{center}
      $\S7.$ \smallПЕРВООБРАЗНАЯ
      \par\noindent\rule{\textwidth}{1.1pt} %the line after the title
  \end{center}
  \par Пример 4.
  \[\int \Big( x+\frac{1}{\sqrt{x}} \Big)^{2}\, dx = \int \Big(x^2+2\sqrt{x}+\frac{1}{x}\Big)\, dx=\]
  \[=\int x^2 \, dx + 2 \int x^{1/2}\, dx+ \int \frac{1}{x}\, dx = \frac{1}{3} x^3 +\frac{4}{3} x^{3/2} +\ln {|x|} + c.   \]
  \par Пример 5.
  \[\int \textrm{cos}^2\, \frac{x}{2}\, dx= \int \frac{1}{2}(1+\textrm{cos}\, x)\, dx=\frac{1}{2}\int (1+\textrm{cos}\, x)\, dx =\]
  \[=\frac{1}{2}\int 1\, dx +\frac{1}{2}\int\textrm{cos}\, x\, dx=\frac{1}{2}x + \frac{1}{2}\, \textrm{sin}\, x+c.\]
  \par\textbf{b. Интегрирование по частям.} Формулу (6) можно переписать в виде
  \[u(x)v(x)=\int u(x)\, dv(x) +\int v(x)\, du(x) +c.    \]
  или, что то же самое, в виде
 \[ \int u(x)\, dv(x)=u(x)v(x) -\int v(x)\, du(x) +c.\eqno (6')\]
\par Это означает, что при отыскании первообразной функции $u(x)v'(x)$ дело можно свести к отысканию первообразной функции $v(x)u'(x)$, перебросив дифференцирование на другой сомножитель и частично проинтегрировав функцию,как показано в $(6')$, выделив при этом член $u(x)v(x)$. Формулу $(6')$ называют формулой итегрирования по частям.
\par Пример 6.
\[\int \ln{x}\, dx = xln{x} -\int x\cdot\frac{1}{x}\, dx=x\ln{x}-\int 1\, dx=\]
\[=x\ln{x}-x+c.\]
\par Пример 7.
\[\int x^2e^{x}\, dx=\int x^2\, de^{x}=x^2e^{x}-\int e^{x}\, dx^2=x^2e^{x}-2\int xe^{x}\, dx=\]
\[=x^2e^x - 2\int x\, de^{x}=x^2e^x - 2\Big(xe^x -2\int e^{x}\, dx\Big)= \]
\[=x^2e^x - 2xe^x+2e^x+c=(x^2 -2x+ 2)e^x +c. \]
\newpage
 \begin{center}
     
     ГЛ. V. ДИФФЕРЕНЦИАЛЬНОЕ ИСЧИСЛЕНИЕ
     \par\noindent\rule{\textwidth}{1.1pt} %the line after the title
     
 \end{center}
\par\textbf{c. Замена переменной в неопределенном интеграле.} Формула (7) показывает, что при отыскании первообразной функции $(f\circ \varphi)(t)\cdot \varphi '(t)$ можно поступать следующим образом
\[\int(f\circ \varphi)(t)\cdot \varphi '(t)\, dt=\int f(\varphi(t))\, d\varphi(t)=\int f(x)\, dx=F(x)+c=F(\varphi(t))+c,\]
т.е. сначала произвести замену $\varphi(t)=x$ под знаком интеграла и перейти к новой переменной $x$, а затем, найдя первообразную как функцию от $x$, вернуться к старой переменной $t$ заменой $x=\varphi(t)$.
\par Пример 8.
\[ \int \frac{t\, dt}{1+t^2} =\frac{1}{2}\int\frac{d(t^2+1)}{1+t^2}= \frac{1}{2}\int \frac{dx}{x}=\frac{1}{2}\ln{|x|}+c=\frac{1}{2}\ln{(t^2+1)}+c. \]
\par Пример 9.
\[\int \frac{dx}{\textrm{sin}\, x}=\int\frac{dx}{2\textrm{sin}\, \frac{x}{2}\textrm{cos}\, \frac{x}{2}}=\int\frac{d(\frac{x}{2})}{\textrm{tg}\, \frac{x}{2}\textrm{cos}^2\, \frac{x}{2}}=\int\frac{du}{\textrm{tg}\, u)\, \textrm{cos}^2\, u}=\]
\[=\int\frac{d(\textrm{tg}\, u)}{\textrm{tg}\, u}=\int\frac{dv}{v}=\ln{|v|}+c=\ln{|\textrm{tg}\, u|}+c=\ln{|\textrm{tg}\, \frac{x}{2}|}+c.\]
\par Мы рассмотрели несколько примеров, в которых использовались порозень свойства a, b, c неопределенного интеграла. На самом деле в большинстве случаев эти свойства используются совместно.
\par Пример 10.
\[\int \textrm{sin}\, 2x\, \textrm{cos}\, 3\, dx=  \frac{1}{2}\int (\textrm{sin}\, 5x - \textrm{sin}\, x)\, dx=\frac{1}{2} \Big(\int\textrm{sin}\, 5x\, dx - \int\textrm{sin}\, x\, dx \Big)= \]
\[=\frac{1}{2} \Big( \frac{1}{5} \int\textrm{sin}\, 5x\, d(5x)+\textrm{cos}\, x  \Big)=  \frac{1}{10} \int\textrm{sin}\, u\, du +\frac{1}{2}\textrm{cos}\, x=-\frac{1}{10}\textrm{cos}\, u+\frac{1}{2}\textrm{cos}\, x+c=\]
\[=\frac{1}{2}\textrm{cos}\, x-\frac{1}{10}\textrm{cos}\, 5x+c.\]
\par Пример 11.
\[\int\textrm{arcsin}\, x\, dx=x\, \textrm{arcsin}\, x - \int x\, d\, \textrm{arcsin}\, x=x\, \textrm{arcsin}\, x - \int\frac{x}{\sqrt{1-x^2 }}dx =  \]
\[=x\, \textrm{arcsin}\, x+ \frac{1}{2} \int\frac{d(1-x^2)}{\sqrt{1-x^2 }}=x\, \textrm{arcsin}\, x+\frac{1}{2} \int u^{-1/2}\, du= \]
\[=x\, \textrm{arcsin}\, x+  u^{1/2} +c= x\, \textrm{arcsin}\, x+  \sqrt{1-x^2 } +c.      \]
\newpage
 \begin{center}
      $\S7.$ \smallПЕРВООБРАЗНАЯ
      \par\noindent\rule{\textwidth}{1.1pt} %the line after the title
  \end{center}
\par Пример 12.
\[\int e^{ax}\, \textrm{cos}\, bx\, dx=\frac{1}{a}\int\textrm{cos}\, bx\, de^{ax}=\frac{1}{a}\, e^{ax}\, \textrm{cos}\, bx  - \frac{1}{a} \int e^{ax}\, d\, \textrm{cos}\, bx =\]
\[=\frac{1}{a}\, e^{ax}\, \textrm{cos}\, bx +\frac{b}{a}\int e^{ax}\, \textrm{sin}\, bx\, dx =\frac{1}{a}\, e^{ax}\, \textrm{cos}\, bx +\frac{b}{a^2}\int\textrm{sin}\, bx\, de^{ax}=\]
\[=\frac{1}{a}\, e^{ax}\, \textrm{cos}\, bx +\frac{b}{a^2}e^{ax}\, \textrm{sin}\, bx-\frac{b}{a^2}\int e^{ax}\, d\, \textrm{sin}\, bx=\]
\[=\frac{a\, \textrm{cos}\, bx + b\, \textrm{sin}\, bx}{a^2}e^{ax} -\frac{a^2}{b^2}\int e^{ax}\, \textrm{cos}\, bx\, dx. \]
Из полученного равенства заключаем, что
\[\int e^{ax}\, \textrm{cos}\, bx\, dx=\frac{a\, \textrm{cos}\, bx + b\, \textrm{sin}\, bx}{a^2+b^2}e^{ax}+c.\]
\par К этому результату можно было бы прийти, воспользовавшись формулой Эйлера и тем обстоятельством, что первообразная функции $e^{(a+ib)x}=e^{ax}\, \textrm{cos}\, bx +ie^{ax}\, \textrm{sin}\, bx $ является функция
\[\frac{1}{a+ib}e^{(a+ib)x} =\frac{a-ib}{a^2+b^2}e^{(a+ib)x}=\frac{a\, \textrm{cos}\, bx + b\, \textrm{sin}\, bx}{a^2+b^2}e^{ax}+i\frac{a\, \textrm{cos}\, bx - b\, \textrm{sin}\, bx}{a^2+b^2}e^{ax}     \]
\end{document}